[En esta parte se escribirá la introducción]

\section{Planteamiento del problema}
El surgimiento de la agilidad ha provocado la necesidad de un nuevo conjunto de herramientas, dado que las metodologías ágiles demandan un enfoque distinto \parencite{mordi:2021}. Las herramientas ágiles se utilizan con mucha frecuencia en las prácticas ágiles de los equipos y organizaciones, la utilización de estas herramientas conlleva a lograr velocidad y eficiencia teniendo un gran impacto en la calidad final del software \parencite{ozkan_mishra:2019}, además proporcionan una variedad de funcionalidades que ayudan a organizar las tareas de desarrollo, así como gestionar el progreso general de los proyectos de software \parencite{alomar:2016}. A pesar de los beneficios evidentes de las herramientas ágiles, existen algunos problemas y desafíos que pueden dificultar su adopción por parte de equipos que no tienen mucha experiencia en el uso de estas, por lo cual es necesario desarrollar alternativas que puedan facilitar la adopción de herramientas de gestión de proyectos ágiles.

La gran mayoría de herramientas se centran en los datos y están diseñados en torno a los formularios, visualmente no son atractivos \parencite{mordi:2021}. El problema se acentúa más cuando los proveedores en su esfuerzo de ofrecer soluciones completas, terminan desarrollando herramientas más grandes y complejas que dificultan su uso \parencite{azizyan:2011, ozkan_mishra:2019}, para los usuarios novatos pueden resultar abrumadoras y excesivamente complicadas \parencite{mordi:2021}. Como demuestra \textcite{dimitrijevic:2015} en su estudio, los usuarios esperan que una herramienta sea sencilla de comprender, fácil de configurar y fácil de usar, todo esto resalta la importancia de la usabilidad como un requisito clave \parencite{mordi:2021}. 

El enfoque empresarial denominado \textit{Freemium}, es ampliamente adoptado por la mayoría de herramientas. En este enfoque, el producto o servicio se ofrece de manera gratuita, pero se cobra por funcionalidades y características adicionales \parencite{ferreira:2018} 
[TEMPORAL NOTE: En segundo término, una gran parte de las herramientas siguen el modelo de negocio "Freemium"... El tercer problema es la falta de comprensión sobre como funcionan las metodologías ágiles y como se integran las herramientas en el proceso] 

En relación a la problemática, pese a que no se encuentren estudios que aborden la adopción de herramientas ágiles, \textcite{alomar:2016} hizo una evaluación exhaustiva de la usabilidad de cuatro herramientas, sus hallazgos podrían inspirar el diseño de herramientas más eficaces. Por otro lado, existen estudios más generales orientados a un análisis sistemático y comparativo entre distintas herramientas con el objetivo de facilitar los procesos de toma de decisiones relacionadas con su selección. En base a lo expuesto, la pregunta que guía toda la investigación es: ¿cuál es el impacto del desarrollo de una herramienta ágil como sistema de soporte y facilitador en la adopción de herramientas ágiles por parte de equipos novatos en el desarrollo de software ágil?

\section{Objetivos de la investigación}
El propósito de la investigación es medir el impacto que tiene el desarrollo de una herramienta de Gestión de Proyectos Ágiles como un sistema de soporte y facilitador para que los equipos o individuos que no tienen suficiente experiencia puedan adoptar y utilizar eficazmente las Herramientas de Gestión de Proyectos Ágiles en el proceso de Desarrollo de Software Ágil.