\section{Paradigma ágil}

  \subsection{Metodologías tradicionales}

  Son conocidos también como metodologías pesadas, fueron los primeros modelos clásicos usados en el desarrollo de software. Se basan principalmente en un enfoque secuencial, el modelo en cascada es el modelo más representativo y una referencia en el desarrollo tradicional de software \cite{islam:2020}. El modelo fue propuesto en 1970, por Winston Royce en su famoso artículo "Managing the Development of Large Software Systems"\footnote{Gestión de Desarrollo de Grandes Sistemas Informáticos}. Muchos interpretaron erróneamente su enfoque como un paradigma de una sola pasada. En realidad, la recomendación de Royce fue realizar el proceso dos veces, un concepto distinto a lo que hoy conocemos como cascada. Resulta paradógico ver la influencia que tuvo el artículo, ya que terminó reforzando un riguroso ciclo de vida secuencial \cite{larman:2003}.

  Por otro lado, \shortciteA{chau:2003} mencionan que las metodologías tradicionales o tayloristas como los denominan, se enfocan demasiado en la documentación como medio principal en la recopilación del conocimiento adquirido durante el ciclo de vida de un proyecto de software. Externalizar el conocimiento de forma explícita explican implica generar numerosos documentos para asegurar los requisitos, diseños, desarrollo, y problemas de gestión. Sin embargo, la mayoría de los conocimientos en ingeniería de software son tácitos, y de los conocimientos explícitos solo unos pocos pueden ser documentados a detalle debido a las limitaciones de tiempo o el enorme esfuerzo que representa para el desarrollador documentar lo que ya sabe. Por eso mismo las metodologías ágiles se enfocan en una documentación mínima.

  \subsection{Orígenes de las metodologías ágiles}
  Aunque muchas personas consideran los métodos ágiles y el desarrollo incremental e iterativo (IID) una alternativa moderna de las metodologías tradicionales, ciertamente sus orígenes y prácticas se remontan a mediados de los años cincuenta \cite{larman:2003}. En febrero del 2001, diecisiete personas lideres en el desarrollo de software decidieron reunirse en Snowbird (Utah) para debatir las similitudes entre distintas metodologías ligeras que existian en ese momento \cite{cockburn:2002}. Todos ellos habían visto fracasar muchos proyectos a causa de los métologías tradicionales, estos estaban enfocados en procesos rígidos, con mecanismos pesados y resultados pobres. A pesar de que fueran sustituidas por implementaciones RUP (Rational Unified Process) tampoco condujeron a resultados mejores \cite{verheyen:2019}.

  En aquel entonces no había una denominación general que englobe estos métodos ligeros, pero se utilizaba el término \textit{ligero} para referirse a ellos, sin embargo, muchos de los implicados no estaban de acuerdo con esta denominación, sentían que no reflejaba adecuadamente la esencia de estos enfoques \cite{fowler:2005,cockburn:2002}. Cada uno de los presentes brindaron su propia perspectiva e interpretación, nadie estaba interesado en fusionar los métodos en una \textit{Metodología Ligera Unificada}, finalmente se termina acuñando el término \textit{ágil} y el nacimiento de la Alianza Ágil, redactándose en el proceso el Manifiesto Ágil con el objetivo de establecer un conjunto de valores y principios para el desarrollo de software \cite{cockburn:2002}.
  \subsection{Manifiesto ágil}

\section{Planificación y estimación ágil}
  \subsection{Planificación ágil}
  \subsection{Estimación ágil}
  \subsection{Plan de lanzamiento o liberación}
  \subsection{Planificación de un sprint}
    \subsubsection{Backlog}
